\documentclass[10pt, a4paper]{article}

\usepackage{times}
\usepackage[left=1.6cm, top=1.5cm, text={18cm, 24.7cm}]{geometry}

\title{Implementační dokumentace k 1. úloze do IPP 2022/2023}
\author{Jméno a příjmení: Denys Petrovskyi\\ Login: xpetro27}
\date{}

\begin{document}

\maketitle
\section{Introduction}

Parse.php provides lexical and syntaxis analysis of the language IPPcode23, which is read from stdin.
The program returns XML code to stdout.

\section{Implementation procedure}
To implement a parser, I followed the next plan:

\begin{itemize}
    \item \textbf{Processing the arguments:} my parse.php supports only one argument which is \verb|--help|,
    this command prints out the usage of parser.
    \item \textbf{Processing the line:} for this purpose a loop is added, which runs until data from stdin ends.
    Each iteration new line of IPPcode23 is being written to the variable \verb|$line|. If on first iteration of the loop
    is not header, the program ends with error, there is a variable for that is called \verb|$header|, it is boolean
    and after header was read it becomes \emph{true}.
    \par After line is read, it gets cleard from all comments by using function \verb|str_split()| and \verb|explode()|.
    Also all the problematic characters in XML \verb|('<', '>', '&')| are being replaced by \verb|('&lt;', '&gt;', '&amp;')|.
    Using the same function \verb|explode()| instruction gets seperated from it arguments.
    \item \textbf{Printing out XML code:} after line is splitted the new object of class \verb|Line| is created.
    By using it methods instructions are being checked for lexical and sytaxis errors with regular expresions and function \verb|preg_match()|
    and then XML code is being printed out. Usage of certain method depends on arguments the instruction has.
    In \verb|switch()| all the instructions are divided into groups to which a certain method belongs.
    After all the input code is processed the program ends.

\end{itemize}

\end{document}